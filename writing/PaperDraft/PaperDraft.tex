\documentclass[]{elsarticle} %review=doublespace preprint=single 5p=2 column
%%% Begin My package additions %%%%%%%%%%%%%%%%%%%
\usepackage[hyphens]{url}
\usepackage{lineno} % add
\providecommand{\tightlist}{%
  \setlength{\itemsep}{0pt}\setlength{\parskip}{0pt}}

\bibliographystyle{elsarticle-harv}
\biboptions{sort&compress} % For natbib
\usepackage{graphicx}
\usepackage{booktabs} % book-quality tables
%% Redefines the elsarticle footer
%\makeatletter
%\def\ps@pprintTitle{%
% \let\@oddhead\@empty
% \let\@evenhead\@empty
% \def\@oddfoot{\it \hfill\today}%
% \let\@evenfoot\@oddfoot}
%\makeatother

% A modified page layout
\textwidth 6.75in
\oddsidemargin -0.15in
\evensidemargin -0.15in
\textheight 9in
\topmargin -0.5in
%%%%%%%%%%%%%%%% end my additions to header

\usepackage[T1]{fontenc}
\usepackage{lmodern}
\usepackage{amssymb,amsmath}
\usepackage{ifxetex,ifluatex}
\usepackage{fixltx2e} % provides \textsubscript
% use upquote if available, for straight quotes in verbatim environments
\IfFileExists{upquote.sty}{\usepackage{upquote}}{}
\ifnum 0\ifxetex 1\fi\ifluatex 1\fi=0 % if pdftex
  \usepackage[utf8]{inputenc}
\else % if luatex or xelatex
  \usepackage{fontspec}
  \ifxetex
    \usepackage{xltxtra,xunicode}
  \fi
  \defaultfontfeatures{Mapping=tex-text,Scale=MatchLowercase}
  \newcommand{\euro}{€}
\fi
% use microtype if available
\IfFileExists{microtype.sty}{\usepackage{microtype}}{}
\ifxetex
  \usepackage[setpagesize=false, % page size defined by xetex
              unicode=false, % unicode breaks when used with xetex
              xetex]{hyperref}
\else
  \usepackage[unicode=true]{hyperref}
\fi
\hypersetup{breaklinks=true,
            bookmarks=true,
            pdfauthor={},
            pdftitle={Tropical storms and associated risks to accidental, cardiovascular, and respiratory mortality in {[}x{]} United States communities, 1988--2005},
            colorlinks=true,
            urlcolor=blue,
            linkcolor=magenta,
            pdfborder={0 0 0}}
\urlstyle{same}  % don't use monospace font for urls
\setlength{\parindent}{0pt}
\setlength{\parskip}{6pt plus 2pt minus 1pt}
\setlength{\emergencystretch}{3em}  % prevent overfull lines
\setcounter{secnumdepth}{0}
% Pandoc toggle for numbering sections (defaults to be off)
\setcounter{secnumdepth}{0}
% Pandoc header


\usepackage[nomarkers]{endfloat}

\begin{document}
\begin{frontmatter}

  \title{Tropical storms and associated risks to accidental, cardiovascular, and
respiratory mortality in {[}x{]} United States communities, 1988--2005}
    \author[Colorado State University]{Meilin Yan}
   \ead{meilin.yan@colostate.edu} 
  
    \author[Colorado State University]{Joshua Ferreri}
   \ead{joshua.m.ferreri@gmail.com} 
  
    \author[NASA Marshall Space Flight Center]{Mohammad Z. Al-Hamdan}
   \ead{mohammad.alhamdan@nasa.gov} 
  
    \author[NASA Marshall Space Flight Center]{William L. Crosson}
   \ead{bcrosson@usra.edu} 
  
    \author[University of Michigan]{Seth Guikema}
   \ead{sguikema@umich.edu} 
  
    \author[Johns Hopkins Bloomberg School of Public Health]{Roger D. Peng}
   \ead{rdpeng@jhu.edu} 
  
    \author[Colorado State University]{G. Brooke Anderson\corref{c1}}
   \ead{brooke.anderson@colostate.edu} 
   \cortext[c1]{Corresponding Author}
      \address[Colorado State University]{Department of Environmental \& Radiological Health Sciences, Lake
Street, Fort Collins, CO, 80525}
    \address[NASA Marshall Space Flight Center]{Universities Space Research Association, 320 Sparkman Dr., Huntsville,
AL, 35805}
    \address[University of Michigan]{Department of Industrial and Operations Engineering, 1205 Beal Ave., Ann
Arbor, MI, 48109}
    \address[Johns Hopkins Bloomberg School of Public Health]{Department of Biostatistics, 615 North Wolfe Street, Baltimore, MD,
21205}
  
  \begin{abstract}
  Here we explore the association between Atlantic basin tropical storms
  and risks of accidental, cardiovascular, and respiratory mortality in
  {[}x{]} United States counties between 1987 and 2005. We investigate
  whether estimated risks differ depending on whether exposure to a storm
  is determined by distance, rainfall, or maximum wind.
  
  We found \ldots{}
  \end{abstract}
  
 \end{frontmatter}

\section{Introduction}\label{introduction}

\paragraph{What we know about tropical storms and mortality
risks}\label{what-we-know-about-tropical-storms-and-mortality-risks}

The East and Gulf Coasts of the United States are commonly exposed to
powerful and destructive tropical cyclonic storms. These storms---which
include hurricanes, tropical storms, and tropical depressions---are
expected to become stronger and more frequent in the Atlantic with
climate change (Seneviratne et al. 2012 -- double-check formatting for
this reference). However, the full scale of health impacts from cyclonic
storms remains poorly characterized. Most epidemiologic studies of
cyclonic storms either identify immediate health needs following a
single storm (rapid needs surveys) (Centers for Disease Control and
Prevention 2004a; Centers for Disease Control and Prevention 2009b;
Greenough et al. 2008; Kessler and Hurricane Katrina Community Advisory
Group and others 2007; Centers for Disease Control and Prevention 1993a;
Centers for Disease Control and Prevention 2005b; Centers for Disease
Control and Prevention 2004b; Centers for Disease Control and Prevention
2006b; Centers for Disease Control and Prevention 2006c; Centers for
Disease Control and Prevention 2002; Centers for Disease Control and
Prevention 2000; Centers for Disease Control and Prevention 2006d) or
generate death and injury tolls based on medical coding (Jani et al.
2006; Brunkard, Namulanda, and Ratard 2008; Combs et al. 1996; Centers
for Disease Control and Prevention 2006a). Results from rapid needs
surveys are difficult to aggregate across storms because survey research
designs are usually tailored for a specific storm and health need (e.g.,
health needs of the elderly in three Florida counties following
Hurricane Charley (Centers for Disease Control and Prevention 2004a)),
and so are of limited used in generating overall estimates of health
risks. Death and injury tolls that are based on medical coding (i.e.,
International Classification of Disease codes of ``disaster-related''
outcomes (World Health Organization 2012 -- improve reference format for
this reference)), are more comprehensive than rapid needs surveys, but
are also difficult to quantitatively aggregate because of large
variation in criteria used by health officials to classify a health
outcome as ``disaster-related'' (Combs et al. 1996; Centers for Disease
Control and Prevention 2006a; Centers for Disease Control and Prevention
1993b; Combs et al. 1999). For example, medical coding of the health
impacts of Hurricane Andrew depended on the separate judgments of 24
different medical examiners in Florida and 64 coroners in Louisiana
(Combs et al. 1996). Finally, disaster assessments based on medical
coding may miss increases in rates of common outcomes, like heart
attacks and respiratory problems, because these commonly occur outside
of disasters and so are harder for a coroner to distinguish on a
case-by-case basis as ``disaster-related'', compared to
well-characterized storm risks like drowning (Centers for Disease
Control and Prevention 2000; Brunkard, Namulanda, and Ratard 2008; Combs
et al. 1996; Centers for Disease Control and Prevention 2006a; Centers
for Disease Control and Prevention 1993b), car accidents (Centers for
Disease Control and Prevention 2006a; Bourque et al. 2007), falls
(Centers for Disease Control and Prevention 2006a), cuts (Bourque et al.
2007), fractures (Bourque et al. 2007), fire injuries (Bourque et al.
2007), electrocutions (Bourque et al. 2007), and carbon monoxide
poisoning (Centers for Disease Control and Prevention 2006a; Bourque et
al. 2007).

There is some evidence that these storms are likely to increase risks of
mortality and morbidity for many causes, including cardiorespiratory
causes. For example, four 2004 Florida hurricanes were associated with
approximately 600 excess deaths (4\% trauma-related; 34\% heart-related;
19\% cancer-related), a mortality impact underestimated by medical
coding by a factor of four (McKinney, Houser, and Meyer-Arendt 2011).
Similarly, emergency department visits increased about 80\% following
Hurricane Katrina in three of the hardest-hit counties in Mississippi
(Centers for Disease Control and Prevention 2006d), and nursing home
mortality rates increased about 70\% in the month after Katrina compared
to baseline rates (Dosa et al. 2010). Many Katrina evacuees arrived at
Louisiana shelters with chest pain, shortness of breath, and dehydration
(Greenough et al. 2008), while evacuees in Texas suffered from
respiratory problems and infections (Mortensen and Dreyfuss 2008).

Tropical storms bring a number of hazards that can cause direct
fatalities. For example, tropical storm-associated tornadoes caused over
300 direct deaths in the United States between 1995 and 2009 (Moore and
Dixon 2012).

One study of all landfalling US hurricanes between 1970 and 2007 found
that direct hurricane fatality counts were directly associated with wind
and rain exposures within a county (Czajkowski, Simmons, and Sutter
2011).

Over 80\% of the direct tropical storm fatalities in the US between 1970
and 1999 were drownings (Rappaport 2000). While storm surge was a major
threat to life from tropical storms in the early 20th century and
earlier, in more recent tropical storms, inland flooding has caused more
direct hurricane fatalities (Rappaport 2000).

Studies have also investigated other, non-fatal health risks associated
with tropical storms in the United States, including adverse birth
outcomes (Currie and Rossin-Slater 2013), autism prevalence (Kinney et
al. 2008), \ldots{} .

The worst tropical storms, in terms of direct fatalities, tend to be
early-in-season storms (Rappaport 2000), which may be because the
conditions that cause slow-moving storms, which can bring dangerous
inland rains and flooding, are more likely in June through August
(Rappaport 2000).

Tropical storm winds can also cause direct fatalities, through
structural damage, falling trees, and wind-borne debris (Rappaport
2000).

\paragraph{Limitations of previous
research}\label{limitations-of-previous-research}

Most research on hurricane-related deaths have focused on direct deaths
(e.g., Czajkowski, Simmons, and Sutter (2011), Rappaport (2000)).
``Direct'' deaths associated with a tropical storm include drownings
(from both storm surge and freshwater flooding) and wind-related deaths
(e.g., from falling trees or structural damage to a building) (Rappaport
2000). Typically, these casualty counts are aggregated based on reports,
and different reports and databases can disagree in the numbers of
casualities for an event (Rappaport 2000). Storm event databases that
rely on reporting, including NOAA's Storm Events database, often provide
casuality estimates that undercount storm-related deaths (Moore and
Dixon 2012). Subjective decisions are often required in creating
tropical storm causality datasets based on storm event databases and
reports (Rappaport 2000).

Indirect hurricane deaths can come from causes like heart attacks, car
accidents, and electrocutions (Rappaport 2000), but are studied much
more rarely than direct hurricane deaths.

\paragraph{Direct casualties versus total mortality
risks}\label{direct-casualties-versus-total-mortality-risks}

\paragraph{Coastal versus inland
risks}\label{coastal-versus-inland-risks}

Tropical storm hazards can exist well inland of where the storm makes
landfall. For example, one study of tropical storm-related wind
exposures found that most states in the eastern United States, even
those well inland of the cost, have experienced severe winds related to
tropical storms or decayed tropical storms at some time between 1900 and
2008 (Kruk et al. 2010). Tropical storms can also cause dangerous inland
flooding, particularly as storms transition to extratropical {[}right
word?{]} systems (Halverson 2015; Atallah and Bosart 2003). Further,
almost half of the wind-related direct hurricane fatalities in the US
between 1970 and 2008 occurred in inland counties, rather than coastal,
counties (Czajkowski, Simmons, and Sutter 2011).

A number of historic storms have had extreme impacts well inland from
the coast, including extreme impacts from rain for Floyd (1999) and
Allison (2001) and extreme impacts from wind for Hugo (1989) and Fran
(1996) (Kruk et al. 2010).

While some studies have found evidence of decreasing direct deaths from
hurricanes over time in the US {[}citations{]}, other studies of more
recent periods have not found a decreasing trend when inland direct
deaths are also included in the analysis (Czajkowski, Simmons, and
Sutter 2011). In present-days periods (e.g., 1970--2007, in one study
(Czajkowski, Simmons, and Sutter 2011)), many direct hurricane
fatalities occur in inland, rather than coastal, counties. This is
particularly true for some storms that have brought dangerous freshwater
flooding (e.g., Floyd in 1999, Allison in 2001, {[}storms{]}
{[}citations{]}) or inland winds (e.g., Hugo in 1989, {[}storms{]}
{[}citations{]}). These storms have not always been particularly strong
at landfall; for example, Allison (2001) was a tropical storm, rather
than hurricane, at landfall {[}citation{]}.

In coastal counties, the number of direct hurricane fatalites between
1970 and 2008 tended to increase at closer distances to the storm's
center, while the inverse was true for inland counties, with more direct
fatalities in counties a bit futher from the storm center track
(Czajkowski, Simmons, and Sutter 2011).

A study of direct tropical storms in the US between 1970 and 1999 found
only about 25\% of these direct deaths were in coastal counties
(Rappaport 2000).

\paragraph{Ways of measuring exposure and how they could influence risk
estimates}\label{ways-of-measuring-exposure-and-how-they-could-influence-risk-estimates}

Some storm hazards are strongly associated with distance both from the
center of the storm and from the coast. For example, hurricane winds
typically decay rapidly after the storm makes landfall (Kruk et al.
2010).

Rains from a tropical storm can cause extreme flooding. This flooding
can be related to a number of health risks, including problems with
water quality following the storm, especially if flooding affects
landfills, wastewater treatment plants, or concentrated animal feedlot
operations (Mallin and Corbett 2006).

Wind speeds of a hurricane are also directly associated with power
outages during the storm {[}citation{]}. Power outages can bring a
number of health-related hazards, both short term (e.g., \ldots{}) and
longer-term, including from water quality problems that can result from
wastewater treatment plants losing power (Mallin and Corbett 2006).

Studies of the health risks of hurricanes have classified storm exposure
in different ways. Some studies have exclusively used distance from the
storm's track (e.g., Currie and Rossin-Slater (2013), \ldots{}), either
distance to some point (Currie and Rossin-Slater 2013) or based on
whether a storm passed within a county's boundaries {[}citations{]}.
Other studies have used metrics that combined distance with other
measures, either to initially assign exposure {[}citations{]} or to
explore effect modification (e.g., Czajkowski, Simmons, and Sutter
(2011) used distance to assign exposure to a storm, then explored
whether rainfall or maximum winds modified risks of direct hurricane
fatalities).

One study found that direct storm deaths were not strongly associated
with the strength of a storm at landfall, since some of the most
dangerous storms could be weaker ones that caused extreme inland
flooding (e.g., Charley in 1998, Alberto in 1994, and Alberto in 1994)
(Rappaport 2000).

In addition to causing drownings, extreme rains from a tropical storm
could create dangerous road conditions (Rappaport 2000), which could
cause or contribute to traffic accidents and could also slow down
emergency response.

A number of storm hazards are directly associated to distance, either
from the storm's center or from the point of landfall, including wind
speed, storm surge, and dangerous waves (Rappaport 2000).

\paragraph{Timing of storms and risks}\label{timing-of-storms-and-risks}

One study found that storms with similar rainfalls cause a more elevated
streamflow, which is related to flooding, if they occur later in the
hurricane season, because of seasonal variations in some of the factors
that modify the relationship between rainfall and flooding risk,
including vegetation patterns and meteorological conditions (Chen,
Kumar, and McGlynn 2015); while one other study also discussed that
flooding risk was reduced early in the hurricane season by the hotter
conditions, which allow more water evaporation (Mallin and Corbett
2006), another study discussed that steering currents tend to be weaker
in the hurricane season, which can cause slower-moving storms and so
more rain and flooding (Rappaport 2000).

\paragraph{What we do in this study}\label{what-we-do-in-this-study}

Here we explore the association between Atlantic basin tropical storms
and risks of accidental, cardiovascular, and respiratory mortality in
{[}x{]} United States counties between 1987 and 2005. We investigate
whether estimated risks differ depending on whether exposure to a storm
is determined by distance, rainfall, or maximum wind.

\section{Data and Methods}\label{data-and-methods}

\paragraph{Hurricane track data}\label{hurricane-track-data}

The HURDAT ``best tracks'' data (Jarvinen and Caso 1978) have been used
in a number of studies, including ones exploring \ldots{},
{[}citations{]} and development of a new hurricane hazard index
(Rezapour and Baldock 2014). The extended best tracks data extension of
HURDAT (Demuth, DeMaria, and Knaff 2006), which we use in this study,
has been used in studies of hurricane-related inland wind exposure (Kruk
et al. 2010), \ldots{} {[}citations{]}.

\paragraph{Rain data}\label{rain-data}

In this study, we use precipitation data from the North American Land
Data Assimilation System, phase 2 (NLDAS-2) (Rui and Mocko 2014) to
characterize rain-based exposure to tropical storms. The NLDAS-2 data
integrates satellite-based and land-based monitoring and applies a
land-surface model to create a reanalysis dataset that is spatially and
temporally complete across the continental United States (Rui and Mocko
2014; Al-Hamdan et al. 2014). This data has been used in previous
research to investigate tropical storms, including in a study of the
relationship between rainfall and streamflow responses for tropical
storms in a North Carolina watershed (Chen, Kumar, and McGlynn 2015).

The NLDAS-2 precipitation data is originally provided hourly for a 1/8
degree grid (Rui and Mocko 2014; Al-Hamdan et al. 2014). To generate
county-level daily rainfall estimates, we first aggregated data at each
grid point, after converting the timestamp of each observation to local
time, to create a daily estimate at the grid point. We then averaged all
grid points within a county's boundaries to generate a county-level
average of daily precipitation (Al-Hamdan et al. 2014). This
county-level precipitation data is publicly available through the US
Centers for Disease Control's Wide-ranging Online Data for
Epidemiological Research (WONDER) database (citation for WONDER;
Al-Hamdan et al. 2014).

\paragraph{Wind data}\label{wind-data}

\paragraph{Health data}\label{health-data}

The mortality data were obtained from the National Center for Health
Statistics and include daily mortality counts, aggregated by community,
for 108 United States communities (1987--2005), including {[}x{]}
communities with exposure based on at least one metric to at least one
tropical storm over the study period ({[}map{]}). For each community, we
obtained daily counts throughout the study period for accidental (ICD
{[}x{]}), cardiovascular (ICD {[}x{]}), and respiratory (ICD {[}x{]})
deaths.

\paragraph{Assigning exposure based on distance-, rain-, and
wind-related
metrics}\label{assigning-exposure-based-on-distance--rain--and-wind-related-metrics}

Tropical storm exposure can be assigned at the city level in a variety
of ways, and models may be sensitive to the method of assignment. For
example, one way to assign exposure is to consider a community
``storm-affected''" on a given day if a storm's track has passed within
a certain distance (e.g., 100 kilometers) of the community's center;
another way is to consider a community ``storm-affected''" if rainfall
surpassed a certain amount (e.g., 75 millimeters) on days within a
certain window of when a storm track passed within a certain distance of
the city center. As part of this research, we investigate the
sensitivity of the models to several reasonable storm exposure metrics.

For each storm, we interpolated tracks from the 6-hour synoptic time
measurements down to 15-minute increments, using a linear interpolation
between observed locations.

To determine the distance between each county and a storm at the storm's
closest approach to the county, we first obtained latitude and longitude
coordinates of each study county's population mean center as of the 2010
US Census {[}citation{]}. We calculated the distance between each county
center and each 15-minute storm location using great circle distance,
assuming an Earth radius of \ldots{} {[}sp citation{]}. We found the
minimum distance of these distance measurements for each study county
for the storm and recorded both the distance and time of this point in
the storm's tracks as the closest distance and closest time for the
county. To enable pairing with rain and health data, we converted the
closest time to local time for the county using timezone designations
based on the Olson / IANA database {[}countytimezone citation, Olson /
IANA database citation{]}. For study communities that are aggregations
of multiple counties (e.g., New York, NY, includes the counties of
\ldots{}), we determined both the mean of all county-specific closest
distances and the minimum of the distance for any of the community's
counties (for single-county communities, these values both equal the
simple closest distance between the county and the storm).

\paragraph{Methods}\label{methods}

By increasing the study power through a national, multi-year analysis,
this study is able to investigate: (1) susceptibility at the individual
level (by age, gender, race) (2) susceptibility at the community level
(by community socioeconomic characteristics); and (3) physical
characteristics that make a storm more dangerous to health (wind speed,
rainfall).

To test differences in susceptibility by individual characteristics, we
modeled overall storm effects with data stratified by age and cause of
death. To test whether community-level characteristics modify average
storm-related health risks, we incorporated community-level
characteristics into the second-level of the pooled two-level normal
independent sampling estimation of the main model results. These
community-level characteristics included measures from the US Census
{[}citation for county data{]}, including unemployment rate and racial
distribution, as well as measures of whether or not a city is coastal
{[}citation, maybe NOAA's ``Spatial Trends in Coastal
Socioeconomics''{]}.

To determine whether certain storm characteristics are associated with
more severe health risks, we estimated the health effects of each
individual storm and then used a Bayesian framework (Everson and Morris
2000) to pool these storm-specific effect estimates while incorporating
three storm characteristics as potential effect modifiers: storm
strength (e.g.~Category 1 Hurricane, Tropical Storm, etc.), total
rainfall within the affected community, and maximum wind speed within
the affected community.

\section{Results}\label{results}

\paragraph{Exposure to tropical storms in study
communities}\label{exposure-to-tropical-storms-in-study-communities}

\section{Discussion}\label{discussion}

\paragraph{Risks from tropical storms}\label{risks-from-tropical-storms}

One study found that hurricanes in Texas between 1996 and 2008 were
significantly associated with some adverse birth outcomes (need for
assisted ventilation for more than 30 minutes, meconium aspiration
syndrome), but not with others (gestation, low birth weight) (Currie and
Rossin-Slater 2013). While this study found that hurricane exposure was
positively associated with risk of some adverse birth outcomes, their
analysis suggested that this association was likely not caused by a
disruption of medical services related to the storm (Currie and
Rossin-Slater 2013). They suggest the possibility that the observed
risks might be associated with increased stress during pregnancy for
women exposed to storms (Currie and Rossin-Slater 2013). Another study
found an increase in autism prevalence associated with exposure to
hurricanes in Louisiana between 1980 and 1995 (Kinney et al. 2008); this
study also hypothesizes that this association might be related to
storm-related stress.

Health risks from tropical storms might be modified by the vulnerability
of the location (e.g., flooding vulnerability of a county) or by
vulnerability factors in the popluation (e.g., high poverty levels)
(Kinney et al. 2008).

\paragraph{Variation in risks based on tropical storm exposure
metric}\label{variation-in-risks-based-on-tropical-storm-exposure-metric}

Different metrics of exposure can identify different storm-related
hazards. For example, slow-moving storms are associated with increased
risk of dangerous rain and flooding (Halverson 2015), but decreased risk
of dangerous inland winds (Kruk et al. 2010).

One study found that a metric of hurricane strength that incorporated
rain, in addition to a storm's intensity as measured by wind speed, was
more strongly associated with the number of {[}direct?{]} deaths and
cost of storm-related damage for US landfalling hurricanes between 2003
and 2012 (excluding Hurricanes Katrina and Sandy) than a metric that
only used wind speed (Rezapour and Baldock 2014).

One study found that, once counties were identified as ``exposed'' based
on a distance-based metric, county-level hurricane direct fatalities
increased as the total rainfall and maximum windspeeds of the storm in
the county increased (Czajkowski, Simmons, and Sutter 2011). However,
this study also found that their distance-based criteria likely excluded
some counties that experienced dangerous exposures and direct hurricane
fatalities (Czajkowski, Simmons, and Sutter 2011).

\paragraph{Risks near the coast versus
inland}\label{risks-near-the-coast-versus-inland}

While some tropical storm hazards (waves, storm surge) occur essentially
exclusively at the coast, other hazards can occur well inland from
landfall. For example, fatal tropical storm-related hurricanes in the US
occur most commonly when a high-intensity storm begins to decay inland
of landfall (Moore and Dixon 2012).

\paragraph{Limitations with distance-based
exposure}\label{limitations-with-distance-based-exposure}

Many of the hazards caused by tropical storms can occur far away from
the storm's center path, including tornadoes, which are most common at
200--500 kilometers from the storm's center (Moore and Dixon 2012).

\paragraph{Limitations with rain-based
exposure}\label{limitations-with-rain-based-exposure}

Storms with the same amount of rainfall can vary substantially in
storm-related streamflow responses (Chen, Kumar, and McGlynn 2015).
Since tropical storm rains likely cause threats to health largely
through flood hazards, a more direct measure of flooding from storms
might provide a more specific measure of when a community experiences a
dangerous exposure to a tropical storm.

Some storms move very slowly, and these tend to be the storms associated
with heavy rainfall (Medlin, Kimball, and Blackwell 2007). It can be
hard to determine an appropriate range of days to include when
calculating storm-related rainfalls. Here, we use a three-day window,
centered on the day when the storm was closest to the county. However,
other studies have used longer periods to try to fully characterize
rainfall for slow-moving storms; for example, one study used a period of
over four days to estimate total rain for Hurricane Danny in 1997
(Medlin, Kimball, and Blackwell 2007).

\paragraph{Limitations with wind-based
exposure}\label{limitations-with-wind-based-exposure}

The wind-based intensity of a hurricane can be strongly associated with
the amount of property damage caused by a hurricane (Pielke and Landsea
1998). However, wind intensity of a storm might not be as strongly
associated with health risks, particularly for indirect deaths.

While many studies of the health impacts of tropical storms have used
wind estimates based on wind models that assumed symmetric decay of
winds on either side of the storm's tracks (e.g., Czajkowski, Simmons,
and Sutter (2011), \ldots{}), here we used wind models that attempted to
capture the asymmetric nature of hurricane winds.

\paragraph{Further research}\label{further-research}

Health risks associated with a cyclonic storm might be aggravated by the
infrastructure damage caused by the storm. Infrastructure damage outside
of storms can bring substantial health risks. For example, 90 excess
deaths were associated with the 2003 Northeastern blackout in New York,
NY, mostly from cardiovascular causes (Anderson and Bell 2012). During a
storm, infrastructure damage may aggravate health risks. Studies of
specific hurricanes have found that infrastructure damage can limit
access to medications, health facilities, food, and water (Centers for
Disease Control and Prevention 2009b; Centers for Disease Control and
Prevention 1993a; Centers for Disease Control and Prevention 2005b;
Centers for Disease Control and Prevention 2004b; Centers for Disease
Control and Prevention 2006d; Smith 1992; Silverman et al. 1995); create
dangerous conditions in hospitals (Brunkard, Namulanda, and Ratard 2008;
Silverman et al. 1995); and cause carbon monoxide poisonings (Centers
for Disease Control and Prevention 2000; Centers for Disease Control and
Prevention 2006d; Bourque et al. 2007; Centers for Disease Control and
Prevention 2005a; Centers for Disease Control and Prevention 2009a).
Flooding from Tropical Storm Allison in 2001 closed or debilitated nine
hospitals in Houston, TX, substantially reducing the city's hospital bed
capacity and overwhelming remaining emergency departments (D'Amore and
Hardin 2005). Hurricane Wilma (2005) caused power outages for about 3
million residences (Centers for Disease Control and Prevention 2006c);
Hurricane Ike (2008) also caused outages for about 3 million residences,
and in parts of Houston service was out for weeks (Centers for Disease
Control and Prevention 2009b). Storm damage can also increase exposure
to environmental health hazards---flooding from Hurricane Floyd closed
over 20 water treatment plants and contaminated local water sources with
untreated hog waste (Setzer and Domino 2004). Future research could
investigate whether storm health risks are aggravated by storm-related
road damage and power outages.

\section*{References}\label{references}
\addcontentsline{toc}{section}{References}

\hypertarget{refs}{}
\hypertarget{ref-AlHamdan2014}{}
Al-Hamdan, Mohammad Z, William L Crosson, Sigrid A Economou, Maurice G
Estes Jr, Sue M Estes, Sarah N Hemmings, Shia T Kent, et al. 2014.
``Environmental Public Health Applications Using Remotely Sensed Data.''
\emph{Geocarto International} 29 (1): 85--98.
doi:\href{https://doi.org/10.1080/10106049.2012.715209}{10.1080/10106049.2012.715209}.

\hypertarget{ref-Anderson2012}{}
Anderson, G Brooke, and Michelle L Bell. 2012. ``Lights Out: Impact of
the August 2003 Power Outage on Mortality in New York, NY.''
\emph{Epidemiology} 23 (2): 189.

\hypertarget{ref-Atallah2003}{}
Atallah, Eyad H., and Lance F. Bosart. 2003. ``The Extratropical
Transition and Precipitation Distribution of Hurricane Floyd (1999).''
\emph{Monthly Weather Review} 131: 1063--81.

\hypertarget{ref-Bourque2007}{}
Bourque, Linda B, Judith M Siegel, Megumi Kano, and Michele M Wood.
2007. ``Morbidity and Mortality Associated with Disasters.'' In
\emph{Handbook of Disaster Research}, 97--112. Springer.

\hypertarget{ref-Brunkard2008}{}
Brunkard, Joan, Gonza Namulanda, and Raoult Ratard. 2008. ``Hurricane
Katrina Deaths, Louisiana, 2005.'' \emph{Disaster Medicine and Public
Health Preparedness} 2 (04): 215--23.

\hypertarget{ref-Centers1993Andrew}{}
Centers for Disease Control and Prevention. 1993a. ``Comprehensive
Assessment of Health Needs 2 Months After Hurricane Andrew--Dade County,
Florida, 1992.'' \emph{MMWR} 42 (22): 434.

\hypertarget{ref-Centers1993Andrew2}{}
---------. 1993b. ``Injuries and Illnesses Related to Hurricane
Andrew--Louisiana, 1992.'' \emph{MMWR} 42 (13): 242--4.

\hypertarget{ref-Centers2000Floyd}{}
---------. 2000. ``Morbidity and Mortality Associated with Hurricane
Floyd--North Carolina, September-October 1999.'' \emph{MMWR} 49 (17):
369.

\hypertarget{ref-Centers2002Allison}{}
---------. 2002. ``Tropical Storm Allison Rapid Needs
Assessment--Houston, Texas, June 2001.'' \emph{MMWR} 51 (17): 365.

\hypertarget{ref-Centers2004Charley}{}
---------. 2004a. ``Rapid Assessment of the Needs and Health Status of
Older Adults After Hurricane Charley--Charlotte, DeSoto, and Hardee
Counties, Florida, August 27-31, 2004.'' \emph{MMWR} 53 (36): 837--40.

\hypertarget{ref-Centers2004Isabel}{}
---------. 2004b. ``Rapid Community Health and Needs Assessments After
Hurricanes Isabel and Charley--North Carolina, 2003-2004.'' \emph{MMWR}
53 (36): 840.

\hypertarget{ref-Centers2005CO}{}
---------. 2005a. ``Carbon Monoxide Poisoning from Hurricane-Associated
Use of Portable Generators--Florida, 2004.'' \emph{MMWR} 54 (28):
697--700.

\hypertarget{ref-Centers2005Four}{}
---------. 2005b. ``Epidemiologic Assessment of the Impact of Four
Hurricanes--Florida, 2004.'' \emph{MMWR} 54 (28): 693--97.

\hypertarget{ref-Centers2006Katrina3}{}
---------. 2006a. ``Mortality Associated with Hurricane Katrina--Florida
and Alabama, August-October 2005.'' \emph{MMWR} 55 (9): 239--42.

\hypertarget{ref-Centers2006Katrina}{}
---------. 2006b. ``Rapid Community Needs Assessment After Hurricane
Katrina--Hancock County, Mississippi, September 14-15, 2005.''
\emph{MMWR} 55 (9): 234--36.

\hypertarget{ref-Centers2006Wilma}{}
---------. 2006c. ``Rapid Needs Assessment of Two Rural Communities
After Hurricane Wilma--Hendry County, Florida, November 1-2, 2005.''
\emph{MMWR} 55 (15): 429--31.

\hypertarget{ref-Centers2006Katrina2}{}
---------. 2006d. ``Surveillance for Illness and Injury After Hurricane
Katrina--three Counties, Mississippi, September 5-October 11, 2005.''
\emph{MMWR} 55 (9): 231--34.

\hypertarget{ref-Centers2009CO}{}
---------. 2009a. ``Carbon Monoxide Exposures After Hurricane
Ike--Texas, September 2008.'' \emph{MMWR} 58 (31): 845--49.

\hypertarget{ref-Centers2009Ike}{}
---------. 2009b. ``Hurricane Ike Rapid Needs Assessment-Houston, Texas,
September 2008.'' \emph{MMWR} 58 (38): 1066--71.

\hypertarget{ref-Chen2015}{}
Chen, Xing, Mukesh Kumar, and Brian L. McGlynn. 2015. ``Variations in
Streamflow Response to Large Hurricane-Season Storms in a Southeastern
U.S. Watershed.'' \emph{Journal of Hydrometeorology} 16: 55--69.
doi:\href{https://doi.org/10.1175/JHM-D-14-0044.1}{10.1175/JHM-D-14-0044.1}.

\hypertarget{ref-Combs1996}{}
Combs, Debra L, R Gibson Parrish, SCOTT JN McNabb, and Joseph H Davis.
1996. ``Deaths Related to Hurricane Andrew in Florida and Louisiana,
1992.'' \emph{International Journal of Epidemiology} 25 (3): 537--44.

\hypertarget{ref-Combs1999}{}
Combs, Debra L, Lynn E Quenemoen, R Gibson Parrish, and Joseph H Davis.
1999. ``Assessing Disaster-Attributed Mortality: Development and
Application of a Definition and Classification Matrix.''
\emph{International Journal of Epidemiology} 28 (6): 1124--9.

\hypertarget{ref-Currie2013}{}
Currie, Janet, and Maya Rossin-Slater. 2013. ``Weathering the Storm:
Hurricanes and Birth Outcomes.'' \emph{Journal of Health Economics} 32:
487--503.
doi:\href{https://doi.org/10.1016/j.jhealeco.2013.01.004}{10.1016/j.jhealeco.2013.01.004}.

\hypertarget{ref-Czajkowski2011}{}
Czajkowski, Jeffrey, Kevin Simmons, and Daniel Sutter. 2011. ``An
Analysis of Coastal and Inland Fatalities in Landfalling US
Hurricanes.'' \emph{Natural Hazards} 59: 1513--31.
doi:\href{https://doi.org/10.1007/s11069-011-9849-x}{10.1007/s11069-011-9849-x}.

\hypertarget{ref-Demuth2006}{}
Demuth, Julie L, Mark DeMaria, and John A Knaff. 2006. ``Improvement of
Advanced Microwave Sounding Unit Tropical Cyclone Intensity and Size
Estimation Algorithms.'' \emph{Journal of Applied Meteorology and
Climatology} 45 (11): 1573--81.

\hypertarget{ref-Dosa2010}{}
Dosa, David, Zhanlian Feng, Kathy Hyer, Lisa M Brown, Kali Thomas, and
Vincent Mor. 2010. ``Effects of Hurricane Katrina on Nursing Facility
Resident Mortality, Hospitalization, and Functional Decline.''
\emph{Disaster Medicine and Public Health Preparedness} 4 (S1):
S28--S32.

\hypertarget{ref-Damore2005}{}
D'Amore, Adanto R, and Charles K Hardin. 2005. ``Air Force Expeditionary
Medical Support Unit at the Houston Floods: Use of a Military Model in
Civilian Disaster Response.'' \emph{Military Medicine} 170 (2): 103.

\hypertarget{ref-Everson2000}{}
Everson, Philip J, and Carl N Morris. 2000. ``Inference for Multivariate
Normal Hierarchical Models.'' \emph{Journal of the Royal Statistical
Society: Series B (Statistical Methodology)} 62 (2): 399--412.

\hypertarget{ref-Greenough2008}{}
Greenough, P Gregg, Michael D Lappi, Edbert B Hsu, Sheri Fink, Yu-Hsiang
Hsieh, Alexander Vu, Clay Heaton, and Thomas D Kirsch. 2008. ``Burden of
Disease and Health Status Among Hurricane Katrina-- Displaced Persons in
Shelters: A Population-Based Cluster Sample.'' \emph{Annals of Emergency
Medicine} 51 (4): 426--32.

\hypertarget{ref-Halverson2015}{}
Halverson, Jeffrey B. 2015. ``Second Wind: The Deadly and Destructive
Inland Phase of East Coast Hurricanes.'' \emph{Weatherwise} 68 (2):
20--27.
doi:\href{https://doi.org/10.1080/00431672.2015.997562}{10.1080/00431672.2015.997562}.

\hypertarget{ref-Jani2006}{}
Jani, Asim A, M Fierro, Sarah Kiser, V Ayala-Simms, DH Darby, S Juenker,
R Storey, Catherine Reynolds, J Marr, and Greg Miller. 2006. ``Hurricane
Isabel-Related Mortality----Virginia, 2003.'' \emph{Journal of Public
Health Management and Practice} 12 (1): 97--102.

\hypertarget{ref-Jarvinen1988}{}
Jarvinen, Brian R., and Eduardo L. Caso. 1978. ``A Tropical Cyclone Data
Tape for the North Atlantic Basin, 1886-1977: Contents, Limitations, and
Uses.'' \emph{NOAA Technical Memorandum NWS NHC 6}.

\hypertarget{ref-Kessler2007}{}
Kessler, Ronald C, and Hurricane Katrina Community Advisory Group and
others. 2007. ``Hurricane Katrina's Impact on the Care of Survivors with
Chronic Medical Conditions.'' \emph{Journal of General Internal
Medicine} 22 (9): 1225--30.

\hypertarget{ref-Kinney2008}{}
Kinney, Dennis K, Andrea M Miller, David J Crowley, and Erika Gerber.
2008. ``Autism Prevalence Following Prenatal Exposure to Hurricanes and
Tropical Storms in Louisiana.'' \emph{Journal of Autism and
Developmental Disorders} 38 (3). Springer: 481--88.

\hypertarget{ref-Kruk2010}{}
Kruk, Michael C., Ethan J. Gibney, David H. Levinson, and Michael
Squires. 2010. ``A Climatology of Inland Winds from Tropical Cyclones
for the Eastern United States.'' \emph{Journal of Applied Meteorology
and Climatology} 49: 1538--47.
doi:\href{https://doi.org/10.1175/2010JAMC2389.1}{10.1175/2010JAMC2389.1}.

\hypertarget{ref-Mallin2006}{}
Mallin, Michael A., and Catherine A. Corbett. 2006. ``How Hurricane
Attributes Determine the Extent of Environmental Effects: Multiple
Hurricanes and Different Coastal Systems.'' \emph{Estuaries and Coasts}
29 (6A): 1046--61.

\hypertarget{ref-Mckinney2011}{}
McKinney, Nathan, Chris Houser, and Klaus Meyer-Arendt. 2011. ``Direct
and Indirect Mortality in Florida During the 2004 Hurricane Season.''
\emph{International Journal of Biometeorology} 55 (4): 533--46.

\hypertarget{ref-Medlin2007}{}
Medlin, Jeffrey M., Sytske K. Kimball, and Keith G. Blackwell. 2007.
``Radar and Rain Gauge Analysis of the Extreme Rainfall During Hurricane
Danny's (1997) Landfall.'' \emph{Monthly Weather Review} 135: 1869--88.

\hypertarget{ref-Moore2012}{}
Moore, Todd W., and Richard W. Dixon. 2012. ``Tropical Cyclone-Tornado
Casualties.'' \emph{Natural Hazards} 61: 621--34.
doi:\href{https://doi.org/10.1007/s11069-011-0050-z}{10.1007/s11069-011-0050-z}.

\hypertarget{ref-Mortensen2008}{}
Mortensen, Karoline, and Zachary Dreyfuss. 2008. ``How Many Walked
Through the Door?: The Effect of Hurricane Katrina Evacuees on Houston
Emergency Departments.'' \emph{Medical Care} 46 (9): 998--1001.

\hypertarget{ref-Pielke1998}{}
Pielke, Roger A., and Christopher W. Landsea. 1998. ``Normalized
Hurricane Damages in the United States: 1925-95.'' \emph{Weather and
Forecasting} 13: 621--31.

\hypertarget{ref-Rappaport2000}{}
Rappaport, Edward N. 2000. ``Loss of Life in the United States
Associated with Recent Atlantic Tropical Cyclones.'' \emph{Bulletin of
the American Meteorological Society} x: 2065--73.

\hypertarget{ref-Rezapour2014}{}
Rezapour, Mehdi, and Tom E. Baldock. 2014. ``Classification of Hurricane
Hazards: The Importance of Rainfall.'' \emph{Weather and Forcasting} 29:
1319--31.

\hypertarget{ref-Rui2014}{}
Rui, Hualan, and David Mocko. 2014. ``Readme Document for North America
Land Data Assimilation System Phase 2 (NLDAS-2) Products.'' In
\emph{Goddard Earth Sciences Data and Information Services Center}.

\hypertarget{ref-Seneviratne2012}{}
Seneviratne, SI, N Nicholls, D Easterling, CM Goodess, S Kanae, J
Kossin, Y Luo, et al. 2012. ``Changes in Climate Extremes and Their
Impacts on the Natural Physical Environment: An Overview of the IPCC
SREX Report.'' In \emph{EGU General Assembly Conference Abstracts},
109--230.

\hypertarget{ref-Setzer2004}{}
Setzer, Christian, and Marisa Elena Domino. 2004. ``Medicaid Outpatient
Utilization for Waterborne Pathogenic Illness Following Hurricane
Floyd.'' \emph{Public Health Reports} 119 (5): 472.

\hypertarget{ref-Silverman1995}{}
Silverman, Michael A, Michael Weston, Maria Llorente, Charles Beber, and
Rosa Tam. 1995. ``Lessons Learned from Hurricane Andrew: Recommendations
for Care of the Elderly in Long-Term Care Facilities.'' \emph{Southern
Medical Journal} 88 (6): 603--8.

\hypertarget{ref-Smith1992}{}
Smith, Keith. 1992. \emph{Environmental Hazards: Assessing Risk and
Reducing Disaster}. Routledge.

\hypertarget{ref-WHO2012}{}
World Health Organization. 2012. ``International Classification of
Diseases (ICD).''

\end{document}


